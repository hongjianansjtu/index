\documentclass{beamer}
\XeTeXlinebreaklocale "zh"
\XeTeXlinebreakskip = 0pt plus 1pt

\usetheme{CambridgeUS}
\usepackage[most]{tcolorbox} 
\usepackage{fontspec}
\usepackage{tikz}
\usepackage{amsmath, amssymb}
\usepackage{hyperref}
\setmainfont{Noto Serif CJK TC}
\setsansfont{Noto Sans CJK TC}
\setbeamertemplate{background}{
    \tikz[overlay, remember picture]\node [opacity=0.2] at(current page.east)[below=2cm, left = 0.2cm]{\includegraphics[width=4cm]{logo.png}};
}

\AtBeginSection[]
{
 \begin{frame}<beamer>
 \frametitle{Outline}
 \tableofcontents[currentsection]
 \end{frame}
}
\AtEndDocument{\begin{frame}
\centering \Huge
                  Q \& A
               \end{frame}
                }
\title{Preliminaries of Cloud Security}
\author{洪佳楠, \href{mailto:hongjn@sjtu.edu.cn}{hongjn@sjtu.edu.cn}}

\begin{document}
\begin{frame}
\titlepage
\end{frame}

\frame{\frametitle{Outline}
\tableofcontents%[pausesections]
}
\section{双线性配对}

\frame{\frametitle{Bilinear Pairing}
A mapping $e$ works on three order-$q$ groups ($\mathbb{G}_1, \mathbb{G}_2, \mathbb{G}_T$) as $$e: \mathbb{G}_1\times \mathbb{G}_2 \rightarrow \mathbb{G}_T$$

\begin{tcolorbox}[enhanced,colframe=white,colback=white, fuzzy halo = 1.3mm with gray]
\begin{enumerate}
\item \textcolor{red}{Bilinearity}. $\forall a, b \in \mathbb{Z}_p^2$, $u, v\in \mathbb{G}_1\times \mathbb{G}_2$: $e(u^a, v^b) = e(u,v)^{ab}$
\item \textcolor{red}{Non-degracy}. $u$ and $v$ are generators of $\mathbb{G}_1$ and $\mathbb{G}_2$: $e(u,v)$ generates $\mathbb{G}_T$.\footnote{$e(u,v)$的所有幂次取遍群里的所有元素}
\item \textcolor{red}{Efficiency}. $e$ is a polynomial-time algorithm for all $u,v$.
\end{enumerate}
\end{tcolorbox}
}
\frame{
	\frametitle{配对中的必要解释}
	在密码算法设计中,一般不需要了解具体用了什么数学工具实现了这种操作。

	\begin{itemize}
		\item 一般来说, $\mathbb{G}_T$是乘法群,但$\mathbb{G}_1, \mathbb{G}_2$有被写成加法群,也有被写成乘法群。
		\item 当写成加法群时,双线性特性写成 $e(aP, bQ) = e(P,Q)^{ab}$
		\item 因为群只有一种运算,写法对方案影响不大
	\end{itemize}
	从安全上看,凡是无法从群元素运算,以及配对运算计算的结果都可认为是无法运算的。在密码学上,也特别定义了一些难题,典型的比如BDH
\begin{tcolorbox}[enhanced,colframe=white,colback=white, fuzzy halo = 1.3mm with gray]
	A BDH tuple is as $(g_1, g_2, g_1^a, g_1^b, g_2^c, e(g_1,g_2)^{abc})$
	\begin{itemize}
		\item BDH tuple is difficult to determine or compute.
	\end{itemize}
\end{tcolorbox}
}
\frame{
	\frametitle{BLS signature: 从CBDH难题产生的签名算法}
	\begin{itemize}
		\item \alert{SK} $SK = x \in \mathbb{Z}_q$
		\item \alert{PK} $PK = g_1^x$
		\item \alert{Sign} $\sigma = H(m)^x$
		\item \alert{Verify} check if $e(g_1, \sigma)= e(PK, H(m))$
	\end{itemize}

\vskip 0.2cm

\alert{NOTE:} 某些配对曲线中,$\mathbb{G}_1$和$\mathbb{G}_2$一致,直接写作$\mathbb{G}$
}
\section{线性秘密共享LSSS}
\frame{\frametitle{Lagrange Interpolation}
Any degree-$t-1$ polynomial $$P(x) = \sum_{i=0}^{t-1} a_i x^i$$
can be reconstructed from $t$ distinct points in the curve \footnote{Instance: a degree-2 curve is determined by 3 points}. 

\begin{tcolorbox}[enhanced,colframe=white,colback=white, fuzzy halo = 1.3mm with gray]
With $t$ points $(x_i, y_i = P(x_i))$
$$P(x) = \sum_{1\leq i\leq t} l_i y_i$$
$$ l_i = \prod_{1\leq j\leq t}^{j\neq i} \frac{x - x_j}{x_i - x_j} $$
\end{tcolorbox}
}
\frame{\frametitle{理解每一项的物理含义}
\begin{itemize}
	\item 首先确认唯一性。在此前提下,任何方法构造出来的多项式都是确定性的。
\end{itemize}

\begin{tcolorbox}[enhanced,colframe=white,colback=white, fuzzy halo = 1.3mm with gray]
	\begin{itemize}
		\item 构造了t个多项式$P_i=w_i\prod_{1\leq j\leq t}^{j\neq i}(x-x_j)$
			\begin{itemize}
				\item $P_i$ 是在除$x_i$外,其他指定点取之为0的多项式;
				\item 那么,对任意$F(x)$,$P_i+F(x)$在其他给定点$x_j\neq x_i$的值$$F(x_j)+P_i(x_j) = F(x_j)$$
			\end{itemize}
		\item 只需调整$w_i$,使$P_i(x_i) = y_i$
			\begin{itemize}
				\item $w_i = y_i /\prod_j (x-x_j)$
			\end{itemize}
		\item $P(x) = \sum P_i$
	\end{itemize}
\end{tcolorbox}
}
\frame{\frametitle{LSSS 的另一种表示方法}
高维空间上选择一个点
\begin{enumerate}
	\item 在$n$维空间中有互不平行的超平面
	\item $n$个超平面相交于一个点
\end{enumerate}

\vskip 1cm

To express a hyperplane in an \alert{n-dimension} span:
\begin{itemize}
	\item a n-dimension vector: $$(e_1, e_2, \dots, e_n)\cdot \vec{x} = c$$
	\item 互不平行的超平面:线性无关向量
\end{itemize}
}

\frame{
	\frametitle{Hyperplane: Continue}
	考虑$n$个线性无关向量穿过一个点$\Omega \in \mathbb{R}^n$
$$(s_1, s_2, \dots, s_n) \gets 
\begin{pmatrix}
	e_{1,1} &  \cdots & e_{1,n} \\
	\vdots & \ddots & \vdots \\
	e_{n,1} & \cdots & e_{n,n}
\end{pmatrix}
\cdot \Omega
$$
\begin{itemize}
	\item $s_i$ 就是隐藏$\Omega$的秘密信息
	\item 如果想从$s_i$恢复$\Omega$,只需要找到逆矩阵
\end{itemize}
}
\frame{
	\frametitle{LSSS的效果}
	\begin{itemize}
		\item \alert{高效性} 只要凑够足够的share,得到秘密的算法是高效的;
		\item \alert{秘密性} 只要没有凑够,那么对秘密的知识量为0
			\begin{itemize}
				\item 对于多项式。确定其他$t-1$个点,修改第$t$个点,秘密值取遍空间;
				\item 对于超平面,确定其他平面,另一个平面的值能让秘密值取遍全空间。
			\end{itemize}
		\item 一般来说,秘密值是多项式的$P(0)$或空间中点的第一维数值。
	\end{itemize}
}
\section{Paillier Cryptosystem}

\frame{\frametitle{Algorithm}
Big primes $p, q$ are securely chosen:
\begin{enumerate}
	\item $n=pq, \lambda = lcm(p-1, q-1)$, select $g\in \mathbb{Z}_{n^2}^*$. 可以为2
	\item $\mu = (L(g^\lambda \mod n^2))^{-1} \mod n$. \alert{L(x) = (x-1)/n}
	\item pk = $n,g$. sk = $\lambda, \mu$
\end{enumerate}

先不考虑私钥格式与解密,先看加密方法:
\begin{enumerate}
	\item 明文$0\leq m\leq n$;
	\item 选择 $r$ 与 $n$ 互质;
	\item 密文 $c = g^m \cdot r^n \mod n^2$
\end{enumerate}
}

\frame{
	\frametitle{Homomorphic Feature}
	\begin{itemize}
		\item Addition: $c_1\cdot c_2 = g^{m_1+m_2} \cdot (r_1\cdot r_2)^n$
		\item Mulitple: $c_1^{m_2} = (g^{m_1})^{m_2}\cdot (r^{m_2})^n$
	\end{itemize}
	
	\vskip 1cm

	获取两个消息之一,但不向对方暴露获取的是那一个:

	Encrypt $b=0/1$ to $c$, and the opponent does some operations.
}

\end{document}
