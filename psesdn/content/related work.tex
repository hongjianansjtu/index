\section{Related Work}
%%multifactor-authentication
\subsection{Multi-factor Authentication}
Multi-factor authentication provides higher level of security by requiring users to provide two or more different factors proving their identity. Based on the type of evidence provided, authentication factors could be divided into categories of knowledge, ownership, and biometric factors\cite{ometov2018multi}. The most widely employed authentication factor include 



 
%%anonymous credential + attribute-based anonymous credential


\subsection{Anonymous Credential}
Camenisch-Lysyanskaya anonymous credentials, a commonly known implementation of signature-based verifiable credential is one answer to the above problems \cite{voelkel2020selectively}\cite{ates2012warning}. By combing signature-based anonymous credential with zero-knowledge proof, the user could prove its identity to the verifier while minimizing data disclosure\cite{morais2019survey}. What the verifier could get is only the result of the cryptographic verification, and it does not have to know the plaintext of the attribute. In this way, the anonymous credential factor provides users with highly reliable privacy preservation and unlinkability. From 2001 to 2004, Camenisch \textit{et al.}\cite{kongsuwan2020anonymous,camenisch2002design} developed an efficient signature scheme for anonymous credential system known as CL signature. The scheme enables the issuer to sign a single signature on blocks of messages (or attributes), and was the prototype to implement the idemix Anonymous Credential System. In 2004, Boneh \textit{et al.}\cite{boneh2004short} developed a short group signature scheme using pairing-based elliptic-curve cryptography, known as BBS signature. Compared to CL signature, the BBS scheme requires shorter keys. The BBS signature was improved to BBS+ by Au \textit{et al.}\cite{au2006constant} in 2006.
%However, simply showing in complete user credential may leads to leakage of user information that is unnecessary for authentication and linkable user behaviours on different application servers\cite{2022worldofAC}. Camenisch-Lysyanskaya anonymous credentials, a commonly known implementation of signature-based verifiable credential is one answer to the above problems \cite{voelkel2020selectively}\cite{ates2012warning}. By combing signature-based anonymous credential with zero-knowledge proof, the user could prove its identity to the verifier while minimizing data disclosure\cite{morais2019survey}. What the verifier could get is only the result of the cryptographic verification, and it does not have to know the plaintext of the attribute. In this way, the anonymous credential factor provides users with highly reliable privacy preservation and unlinkability. From 2001 to 2004, Camenisch \textit{et al.}\cite{kongsuwan2020anonymous,camenisch2002design} developed an efficient signature scheme for anonymous credential system known as CL signature. The scheme enables the issuer to sign a single signature on blocks of messages (or attributes), and was the prototype to implement the idemix Anonymous Credential System. In 2004, Boneh \textit{et al.}\cite{boneh2004short} developed a short group signature scheme using pairing-based elliptic-curve cryptography, known as BBS signature. Compared to CL signature, the BBS scheme requires shorter keys. The BBS signature was improved to BBS+ by Au \textit{et al.}\cite{au2006constant} in 2006.

%According to the COMMISSION IMPLEMENTING REGULATION, to achieve substantial assurance level, at least two authentication factors from different categories should be employed\cite{commission2015official}. The possession-based authentication require the user to possess a specific piece of information or device known belong to the correct user before they can be granted access to the system\cite{devops2022}, and some typical examples include smart cards\cite{chien2002efficient}, hardware tokens\cite{brown2004use}, etc. However, a major problem of the authentication based on "something the user possesses" is that the user must carry its possession factor around, practically all the time. This has made mobile phone-based authentication an better alternative to dedicated physical devices, since mobile devices are usually carried all the time\cite{multi2022}. Various mobile phone-based multi-factor authentication has been proposed. In the work of Aloul \textit{et al.}\cite{aloul2009multi}, an implementation of 2-factor authentication based on mobile phone is introduced, where the authentication process is done by verifying a one-time password (OTP) generated by an application installed on mobile phone based on attributes unique to both the user and mobile device. In 2010, Google released a 'Google Authenticator' application for generation of OTP codes\cite{kincaid2010google}. However, it requires extra efforts for service applications to support it, and additional configurations for all applications on user's mobile, when user swap devices, such configurations must be done again for all registered applications, and the authenticator application itself is not protected by passcode or biometric lock, therefore it is vulnerable to malware attacks. Another commonly employed mobile phone-based identity authentication is to use the SMS verification code, where the dynamic password used for authentication is generated by the application server and relayed to user's mobile phone through the mobile network operators\cite{li2017mobile}. One major concern about SMS-based authentication is that SMS text is sent in clear text and is at risk of being intercepted. In 5G, the SMS over NAS is ciphered and integrity-protected using the NAS security context built through a 5G AKA process\cite{3GPPTS33501}.



 %%sin
