\section{Introduction}

%% Background of target technology (MFA or authentication, or cryptographic authentication, or anonymous credential)
\IEEEPARstart{T}{he} advances in network technologies from the architecture (e.g., 5th-generation network, Internet of Things) has resulted in the continuous growth in the numbers of connected smart devices through the Internet, which brings in larger and larger market for organizations and enterprises to provide up-layer network services. As these services trend to offer more valuable or private information data (such as personal balance in a bank, private status of the owner's house), and some services are critical to user's safety (like payment), thus the service should be offered with a well-deployed authentication method to resist various adversaries.
%Large amount of data, possibly relevant to every area of human activity, both public and private, will be produced, communicated, gathered, stored, and processed. To keep the transmitted data safe, authentication acts as the first line of defence against adversaries.

According to [], through the process of authentication, a security system validates whether a user could be granted the corresponding access privilege. Based on the authentication mechanism used, the user provides pieces of evidence (factors) of identity so that the security system is convinced whether the user is who it claims to be, or belongs to a certain legitimate user group. Depend on the specific evidence submitted by the user, authentication factors could be categorized into three groups: knowledge factors, ownership factors, and inherent factors. The knowledge factor is something that only the user knows, such as its password or secret key. The possession factor is something that a user has, such as smart cards or mobile phones. The inherent factor refers to something that qualifies a user, such as biometric data. 
%The location factor refers to somewhere the user is, such as current location or time information. 
In some cases, authentication mechanisms could be integrated with cryptography-defined concepts to expand various functions. For example, the cryptographic challenge-response based authentication protocol is an improvement to privacy of the basic password authentication protocol in terms of brute force and dictionary attacks; By employing attribute-based cryptographic framework, one can realize fine-grained access control of data and resources; The anonymous credential system provides privacy-preservation for user attributes and behaviors and unlinkability under a single sign-on framework. 

``Authentication technologies can both advance and undermine privacy interests'' (National Research Council. 2003). An authentication system may require users to reveal identifying personal information to prove their validity, and the act of authentication itself may be recorded and linked with other user behaviors, causing privacy concerns \cite{national2003goes}. Instead of revealing identifiable information, privacy-concerned users may hope to finish authentication step with a service provider only by showing user's belonging to a valid group or by proving its possession of some specific attribute identifiers\cite{holt2002selective}. Anonymous credential system (ACS) is an effective solution to problem of linkability. Users are known to service providers by pseudonyms in such systems, which ensures the anonymity and unlinkability of authentication[]. In addition, ACS enables privacy-preserving single sign-on (SSO)[]. In traditional SSO, a third party identity provider (IdP) serves as single point of failure; And users are at risk of having their activities tracked by the IdP. On contrary, the process of user authentication no longer depends on a single trusted third party in privacy-preserving SSO[]. In []'s work, the anonymous credential system (ACS) is extended to an attribute-based anonymous credential system (ABACS). Compared to ACS, ABACS enables flexible attribute-based access control and minimal disclosure of information, and is thus welcomed by privacy-concerned users.

Key-based (cryptographic) authentication exhibits fragility in the face of key compromise attacks. Multi-factor authentication overcomes this risk. By requiring more than one factor to authenticate a user, the multi-factor authentication (MFA) makes the intruding process more difficult for hackers, thus providing additional layers of security. For example, MFA used on top of username/password authentication may require users to take an extra step such as entering a code sent to a physical device. Even if the perpetrator brute forces the password, it is not able to get pass the second authentication step without possession of the device. Up till now, few researches has been made on the implementation of ABACS with MFA. The popularization of smartphones has lead to increased usage of short message service (SMS) based multi-factor authentication. In these days, most internet applications require users to accomplish the SMS authentication step to get registered or authenticated, making combination user-friendly. However, one concern is that SMS authentication requires users to transmit their personal information to the service provider, including the user's mobile phone number, which identifies a user's identity better than its real name in today's world. Besides, using the same phone number for each authentication risks damaging the unlinkability provided by ABACS. 

\subsection{Our Contribution}

To address the above problems, we construct a multi-factor authentication scheme that combine attribute-based anonymous credential system with SMS-verification code authentication which employs virtual mobile phone number to protect user privacy. We alter the 5G signalling so that a user who has registered for an SMS-relay service at the 5G system network with a virtual mobile phone number could receive the SMS-verification code on its own mobile phone. When the user wants to authenticate with a service provider (SP), it sends to SP information unique to the user with the mobile phone number replaced by a virtual one. Subsequently, the user's phone number is never transmitted onto the internet.

Our scheme realizes an asynchronous SSO system based on an attribute-based anonymous credential framework. The generation of the material for authentication (the attribute-based credential and related information) is decoupled in time with the process of user's authentication at service provider. Hence relying party (the service provider) no longer has to visit the IdP for identity verification. The IdP could be offline while the authentication process running smoothly, and the tracking of user's activities by IdP is prohibited. The ABACS and the employment of virtual mobile phone number provides solid anonymity and unlinkability.

For many online services, intra-service unlinkability is undesirable. If the generation and registration of virtual phone number is arbitrary, a user may have unlimited number of pseudonyms without restriction. 
In our scheme, the generation of virtual mobile phone number and its registration at the 5G system network are designed to be bounded by the unique identifier of the target service provider.
The user could generate multiple virtual numbers once it has acquired an attribute-based credential, but could not generate and register more than one for the same service provider. 




% If a user consistently uses the same identifier, whether it is a real identity or an anonymous one, to participate in transactions with multiple service providers, it risks exposing its behavioral privacy to the outside. The protection of behavioral privacy has become an important topic in information security. In the age of big data, it is technically possible to accurately analyze users' preferences and interests after mining a large amount of users' continuous behaviors, and to predict the future behavior of users with high accuracy. More dangerously, if an adversary already has acquired a part of the behavior information of an observed entity, and compares it with the behavior data of a de-identified identity, the identity of all behaviors of the user may be recovered, which means that the identity privacy of users would be affected to a certain extent. A core requirement to protect behavioral privacy is to achieve unlinkability in multiple user authentications. That is, if two authentication requests are collected, there is no way to tell whether the two requests are from the same user. In our scheme, inter-service unlinkability is desired. The cancellation of interaction with IdP in the authentication process prevents the IdP from tracking user's activities and different RPs. In addition to this, we hope that the correlation of user's authentication with different RPs should be impossible for any other entities in the system. This implies the unlinkability between user's generated random mobile phone numbers for differents RPs and the unlinkability between the assertion constructed by the user for different RPs.


%Network technologies have been advancing rapidly in recent years from the architecture to the up-layer services, such as fifth-generation (5G) radio network and Internet of Things (IoT). The service-based architecture of 5G network and network slicing technique make it easy to converge high-throughput MBB, low-delay IoT and "ultra-reliable low latency communications (URLLC)" into one system, which significantly increase the development of network-dependent technologies\cite{zanella2014internet}. The number of connected devices in the world is expected to exceed 27 billion by 2025\cite{numberconnectediotdevices}. While the paradigm fosters services of convenience in various domains, it should be noted that the pervasive nature of the information sources means that a large amount of data, possibly relevant to every area of human activity, both public and private, will be produced, communicated, gathered, stored, and processed\cite{bonetto2012secure}. In this context, network  security problems will bring about greater threats and heavier losses, especially for organizations operating in healthcare, finance, etc\cite{katrenko2022IoT}. To protect against threats from internet adversaries, authentication has become a necessity. Through authentication, entities using online communication could verify whether the entity they are interacting with is honest or not. In doing this, authentication assures secure systems, secure processes and enterprise information security\cite{5shacklett2022authentication}. 
%To get authenticated, the common practice is that the user provides some sort of login credential to an authentication process for it to decide whether to grant of deny user's access. The credential supplied by the user is what is called the authentication factor\cite{6sharmasecurity}. Basically, authentication factors can be divided into the following three categories: 1) knowledge factor (something you know, e.g. password), 2) possession factor (something you have, e.g. smart cards), 3) inherent factor (something associated with who you are, e.g. fingerprints, retina, voice). Other methods involve attribute-based authentication based on information about users and location based factors\cite{6sharmasecurity}\cite{7alqahtanitechnical}. Examples of traditional authentication technologies include username-password challenge, Authentication And Key Agreement (AKA) protocol, Public Key Infrastructure (PKI) -based authentication, and internet security protocols such as IPsec, TLS, etc. Newly emerging or under-research methods include Identity-based Cryptography (IBC) -based authentication and biometric authentication such as fingerprint, iris scan, etc. Among these methods, cryptographic authentication (also known as Key-based authentication, uses cryptographic keys in a challenge-response handshake to prove one's identity\cite{8keybasedauthentication2022}) is capable of providing various services based on different usage scenarios, e.g. fine-grained access control and privacy-preservation by employing anonymous credentials and attribute-based encryption\cite{9goyal2006attribute, 10guo2014anonymous}. However, key-based authentication exhibits fragility in the face of key compromise attacks\cite{solomon2021how}. To tackle the problem, Multi-factor authentication is called on to provide a boost to security: it is stated that MFA can block over 99.9 percent of account compromise attacks\cite{11onesimpleaction2019}. MFA makes the intruding process more difficult for hackers by providing additional layers of security. For example, MFA used on top of username/password authentication may require users to take an extra step such as entering a code sent to a physical device. Even if the perpetrator brute forces the password, it is not able to get pass the second authentication step without possession of the device.

%''Authentication technologies can both advance and undermine privacy interests" (National Research Council. 2003). An authentication system may require users to reveal identifying personal information to prove their validity, and the act of authentication itself may be recorded and linked with other user behaviors, causing privacy concerns\cite{national2003goes}. Instead of revealing identifiable information, users hope to finish authentication step with a service provider only by showing user's belonging to a valid group or by proving its possession of some specific attribute identifiers\cite{holt2002selective}. Anonymous credential credential is an effective solution to these requirements. With the help of zero-knowledge proof, attribute-based authentication and access control between user and service provider could be achieved while minimizing personal information release\cite{camenisch2002design}. 

%A drawback of anonymous credential is that potential leakage of credentials may lead to identity misuse, which makes anonymous credentials unsuitable for real e-commerce applications and web services that require high security level and strong identifiers\cite{bertino2008security}. Such problem could be addressed by combining anonymous credential with other authentication factors to provide multi-factor authentication. However, the other authentication factor should be chosen with caution. First, a cyptography-based authentication factor is not recommended, as it faces the same drawback as that of the anonymous credential. Such combination of authentication factors has no effect in promoting each other in different security aspects. Second, as has mentioned in the previous paragraph, trivial authentication factors may lead to a breach of privacy protection provided by anonymous credential.


%The analysis of authentication and security protocols involves thorough debugging of the protocol to prove that no flaws could be abused by adversaries or dishonest users to obtain or alter vital information. Such analysis is only possible by using formal techniques\cite{mao1993towards}. Various formal analysis tools have been developed for modeling and performing mathematical analysis on cryptography-based security protocols, such as Scyther, ProVerif, Tamarin-Prover, etc\cite{cremers2008scyther, blanchet2018proverif, meier2013tamarin}. 


%\section{Our contribution}
%In this paper, we present a two-factor credential-based authentication scheme. Our scheme provides strong privacy protection and fine-grained access control. Users could choose to disclose only a part of the attributes in the credential to the relying party, and the relying party could define different access policies requiring different combination of attributes. Our authentication scheme improves the security level by employing a second mobile phone-based authentication factor. By sending to RP a verification code previous received through SMS, the user is able to prove its possession of the mobile phone, which is simple for operation and effective. A virtual mobile phone number generation method is designed with an SMS relying service on the 5G core such that users could hide their real mobile phone number from the internet. The unlinkability of the authentication process is also ensured. Our scheme also features asynchronous single sign-on. The generation and issuance of user's anonymous credential at the identity provider (IdP) is time-decoupled from the user authentication procedure, therefore the IdP needn't always be online. Our scheme realizes a real-name system, in which a user could not cheat the RP into believing it if it has been marked as malicious. In the end, we provide a formal security proof using the tool for formal proof Tamarin-prover to demonstrate that the proposed authentication scheme indeed enforces its security guarantees.