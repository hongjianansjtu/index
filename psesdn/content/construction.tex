\section{Construction}
In this section, we go through an entire authentication process to show the whole flow path of our protocol. To get authenticated, the system goes through four steps: of the initial system setup, the SMS-service registration, the credential issuance, and 2-factor authentication. The notions used for the protocol description are used in table 1.
%moved here for pagination purposes
\begin{table}[!ht]
\begin{center}
\caption{Notions Used For Protocol Description.}
\label{tab1}
\begin{tabular*}{\columnwidth}{|c|p{5.45cm}|}
\hline
\textsc{Symbol} & \centerline{\textsc{Description}}\\
\hline
$\mathrm{S}$ & Credential schema defined by identity provider, defines the number of attributes aggregated in the credential and their corresponding meanings.\\
$n$ & Attribute dimension of the credential issued by identity provider.\\
$\left\{M_{i}\right\}_{i \in[[1, n]]}$ &
Attribute messages sent by the user to the IdP to acquire its credential.
\\
$\left\{m_{i}\right\}_{i \in[[1, n]]}$ &
Attribute values of user converted from $\left\{M_{i}\right\}_{i \in[[1, n]]}$ so that $m_i \in \mathbb{Z}_p$
\\
$t\in [[1, n]]$& The index of the attribute that corresponds to user's real mobile phone number. \\
$M_t(MSISDN), m_t$ & User's real mobile phone number and the corresponding attribute value.\\
$pm$ & Virtual mobile phone number generated by the user for a unique 
relying party.\\
$H_1, H_2, H_3, H_4$ & Four secure cryptographic hash functions that the whole system agrees on.\\
$RID$ & Unique identifier assigned to relying party.\\
$A_d, A_h$ & Set of attribute indexes to be disclosed/hidden, where $A_d$ is defined by relying party's access policy.\\
$\mathcal{M}_d, \mathcal{M}_h$ & Set of user's attribute values to be disclosed/hidden. \\
\hline 
\end{tabular*}
\end{center}
\end{table}
\subsection{System Setup}
In the setup phase, the following algorithms are executed in order only once for the bootstrapping of the identity provider. 
\begin{list}{}{}
    \item{\bf{Setup($1^{\lambda}$)$\rightarrow$($para$): }} Output \textsf{Cred.Setup($1^\lambda$)}.\\
    \item{\bf{KeyGen($para$)$\rightarrow$($sk, pk$): }}Output \textsf{Cred.KeyGen($para$)}. Run by the IdP to generate its private-public key pair $(sk, pk)$ from $para$.\\    
\end{list}
At the end of the setup phase, the public parameters ($para, pk$) of the identity provider should be generated and published to whole system. Apart from that, any relying party that has joined the system should be assign a unique identifier $RID$ and publish its access policy, which then defines $A_d$.
\subsection{Credential Issuance}
The following algorithm is executed periodically by the issuer, when a new user acquires a credential from the identity provider (issuer).
\begin{list}{}{}
    \item{\bf{Issue($pp, sk, \mathcal{M}$)$\rightarrow$($\sigma$): }} Output \textsf{Cred.Issue($pp, sk, \mathcal{M}$)}. The interactive algorithm is executed between the user who wants to acquire an attribute-based anonymous credential and the credential issuer. As a result of the execution, the credential $\sigma$, embedded with the user attributes $\mathcal{M}$, is sent to user through a secure channel.
\end{list}
\subsection{SMS-relay registration}
We describe the algorithm implementing the SMS-relay service registration phase; the algorithms are executed each time a user tries to authenticate with and get access to a new service provider.
\begin{list}{}{}
    \item{\bf{Prove1($H_1, H_2, H_3, p, RID, m_t, s$)$\rightarrow$($\Theta_1$): }}
    pick two randoms $r_t$ and $r_0$ from $\mathbb{Z}_p$ and compute $$R\rightarrow H_1(RID)^{r_t}\cdot H_2(RID)^{r_0}.$$
    
    Run by the user to construct a proof $\Theta_1$ proving that it knows the real mobile phone number $m_t$ and the secret parameter $s$, and that the virtual number $pm$ is generated based on $(RID, m_t, s)$.
\end{list}
\subsection{Two-Factor Authentication}
